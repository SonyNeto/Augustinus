\documentclass[11pt, a4paper]{article}

% Pacotes essenciais para documentos modernos
\usepackage[utf8]{inputenc} % Permite caracteres acentuados diretamente no arquivo .tex
\usepackage[T1]{fontenc}    % Otimiza a renderização de fontes, incluindo caracteres acentuados
\usepackage[portuguese]{babel} % Suporte para idioma português (hifenização, nomes de seções, etc.)
\usepackage[autocompile]{gregoriotex} % Pacote para notação gregoriana
\usepackage[margin=16mm]{geometry} % Define margens (aqui, 1 polegada em todos os lados)

% Informações do documento
\title{Códigos dos Refrãos Gregorianos}
\author{Para Giovanni}
\date{\today} % Usa a data atual ou pode ser uma data específica, por exemplo, \date{Setembro de 2025}


\gresetinitiallines{0}
\begin{document}

% Gera a página de título
\maketitle

\thispagestyle{empty} % Remove o número da página da primeira página, se desejar.

\clearpage % Inicia o conteúdo em uma nova página após o título e o resumo, se houver.

\section*{Refrãos por Modo}
\addcontentsline{toc}{section}{Refrãos por Modo} % Adiciona esta seção não numerada ao índice

Este é o conjunto de códigos para os refrãos, separados de acordo com seus modos, para o Giovanni aplicar no aplicativo.

\subsection*{I MODO}
\addcontentsline{toc}{subsection}{I MODO} % Adiciona este modo ao índice
\gregorioscore{1.gabc}

\subsection*{II MODO}
\addcontentsline{toc}{subsection}{II MODO}
\gregorioscore{2.gabc}

\subsection*{III MODO}
\addcontentsline{toc}{subsection}{III MODO}
\gregorioscore{3.gabc}

\subsection*{IV MODO}
\addcontentsline{toc}{subsection}{IV MODO}
\gregorioscore{4.gabc}

\subsection*{V MODO}
\addcontentsline{toc}{subsection}{V MODO}
\gregorioscore{5.gabc}

\subsection*{VI MODO}
\addcontentsline{toc}{subsection}{VI MODO}
\gregorioscore{6.gabc}

\subsection*{VII MODO}
\addcontentsline{toc}{subsection}{VII MODO}
\gregorioscore{7.gabc}

\subsection*{VIII MODO}
\addcontentsline{toc}{subsection}{VIII MODO}
\gregorioscore{8.gabc}

\end{document}